% ---------- Titelblad Masterproef Faculteit Wetenschappen -----------
% Dit document is opgesteld voor compilatie met pdflatex.  Indien je
% wilt compileren met latex naar dvi/ps, dien je de figuren naar
% (e)ps-formaat om te zetten.
%                           -- december 2012
% -------------------------------------------------------------------
\RequirePackage{fix-cm}
\documentclass[12pt,a4paper,oneside]{article}

% --------------------- In te laden pakketten -----------------------
% Deze kan je eventueel toevoegen aan de pakketten die je al inlaadt
% als je dit titelblad integreert met de rest van thesis.
% -------------------------------------------------------------------
\usepackage{graphicx,xcolor,textpos}
\usepackage{helvet}

% -------------------- Pagina-instellingen --------------------------
% Indien je deze wijzigt, zal het titelblad ook wijzigen.  Dit dien je
% dan manueel aan te passen.
% --------------------------------------------------------------------

\topmargin -10mm
\textwidth 160truemm
\textheight 240truemm
\oddsidemargin 0mm
\evensidemargin 0mm

% ------------------- textpos-instellingen ---------------------------
% Enkele andere instellingen voor het voorblad.
% --------------------------------------------------------------------

\definecolor{green}{RGB}{172,196,0}
\definecolor{bluetitle}{RGB}{29,141,176}
\definecolor{blueaff}{RGB}{0,0,128}
\definecolor{blueline}{RGB}{82,189,236}
\setlength{\TPHorizModule}{1mm}
\setlength{\TPVertModule}{1mm}



%----------------------- Custom stuff -------------------------------

\graphicspath{./}
\usepackage{makeidx}
\index{hoofd}
\makeindex
\usepackage{amsmath}
\usepackage[dutch]{babel}

\usepackage{hyperref}
\usepackage{graphicx}
\usepackage{caption}
\usepackage{subcaption}




%------------------------ Plot packages ----------------------------
\usepackage{tikz}
\usepackage{pgfplots}

\usepackage{pgf}
\usepackage{units}
\usepackage{metalogo}







% --------------------------------------------------------------------

% ------------------ Packages Matthias !!!!!!!!!!!!!!!!!!!!!!!!!!!!!!!!!!!!!!!!!!!!!!!!!!!

\usepackage{placeins}


