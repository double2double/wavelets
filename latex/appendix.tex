


\begin{appendices}

In de appendix bevind zich een deel van de code die gebruikt werd voor het bekomen van bovenstaande resultaten.
De bijgevoegde code is tot een minimum gehouden, enkel de gebruikte algoritme werden toegevoegd en een script om telkens hun gebruik te illustreren.
Het volledige project, inclusief de matlab bestanden om elk van bovenstaande resultaten te genereren kan gevonden worden op \href{https://github.com/double2double/wavelets}{https://github.com/double2double/wavelets}




\section*{Matlab Code}


\subsection*{Code voor ruisreductie}

\lstinputlisting{../src/matlab/den_image.m}
\subsubsection*{Voorbeeld code voor gebruik van denoising.}
In het onderstaande script wordt aan de hand van \verb|fminunc| de optimale waarden voor de therhold waarden bepaald.
De cost functie die wordt gebruikt is de euclidische afstand tussen de originele en de gefilterde afbeelding.
\lstinputlisting{../src/matlab/example_denoising.m}

\subsection*{Code voor inpainting}
\lstinputlisting{../src/matlab/inpainting_fun.m}
\subsection*{Voorbeeld code voor gebruik van inpainting.}
In onderstaand script word een afbeelding ingelezen en beschadigd met witte ruis. Nadien word het inpainting algoritme gebruikt om de ontbrekende pixelwaarden te schatten.
\lstinputlisting{../src/matlab/example_inpainting.m}


\end{appendices}